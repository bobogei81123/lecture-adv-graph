% XeTeX template for IOIcamp printables

% 設定英文字型,不設的話就會使用預設的字型
%\setmainfont[BoldFont=Helvetica]{Times New Roman}  % Changed by Skyly (01/09)
\newfontfamily{\couriernew}{Source Code Pro}
\newfontfamily{\helvetica}{TeX Gyre Adventor}
\newfontfamily{\helveticalight}[BoldFont=TeX Gyre Adventor]{TeX Gyre Adventor}
\newCJKfontfamily{\LiHeiPro}{cwTeX Q Ming}
%\newfontfamily{\Garamond}{Garamond}
\newCJKfontfamily{\Kai}{cwTeX Q Kai}
\newCJKfontfamily{\BookHead}{cwTeX Q Hei}
\DeclareMathVersion{mymath}
\DeclareSymbolFont {myoperators}{OT1}{ccr}{m}{n}
\SetSymbolFont{operators}{mymath}{OT1}{ccr}{m}{n}
\DeclareSymbolFont{myletters}{OML}{ccm}{m}{it}
\SetSymbolFont{letters}{mymath}{OML}{ccm}{m}{it}
\DeclareSymbolFont{mysymbols}{OMS}{ccsy}{m}{n}
\SetSymbolFont{symbols}{mymath}{OMS}{ccsy}{m}{n}
\DeclareSymbolFont{mylargesymbols}{OMX}{ccex}{m}{n}
\SetSymbolFont{largesymbols}{mymath}{OMX}{ccex}{m}{n}
\SetMathAlphabet\mathbf{mymath}{OT1}{ccr}{bx}{n}
\SetMathAlphabet\mathit{mymath}{OT1}{ccr}{m}{it}

% 中文自動換行
\XeTeXlinebreaklocale "zh"
% 文字的彈性間距
\XeTeXlinebreakskip = 0pt plus 1pt

% 段距
\setlength{\parskip}{1em}
% 行距
\linespread{1.2}\selectfont
% 不縮排
\setlength{\parindent}{0pt}

% Wildcard-tmt Title Font (WTF)
\newcommand{\tmtTitleFont}{\bfseries\helveticalight\LiHeiPro}

% 頁首、頁尾
\usepackage{fancyhdr}
\usepackage{calc}
\usepackage[explicit]{titlesec}
\usepackage[usenames,dvipsnames]{xcolor}
\usepackage{pbox}
\pagestyle{fancy}
\setlength{\headheight}{15pt}
%\fancyheadoffset[LE,RO]{\marginparsep+\marginparwidth}
\renewcommand{\chaptermark}[1]{\markboth{\chaptername\ \thechapter.\ #1}{}}
\renewcommand{\sectionmark}[1]{\markright{String}}
\renewcommand{\headrulewidth}{0pt} % and the line
\fancyhf{}
\fancyhead[LE]{\hspace*{-21pt}{\tmtTitleFont\thepage\ \ \ }\begin{tikzpicture}[remember picture, overlay]\fill[fill=black](0,-3pt)rectangle(3pt,0.8\baselineskip);\end{tikzpicture}}
\fancyhead[RO]{\begin{tikzpicture}[remember picture, overlay]\fill[fill=black](0,-3pt)rectangle(3pt,0.8\baselineskip);\end{tikzpicture}\tmtTitleFont{\ \ \ \thepage}}
\fancyhead[LO]{\hspace*{-21pt}{\tmtTitleFont\BookHead\rightmark}}
\fancyhead[RE]{{\tmtTitleFont\BookHead\leftmark}}
\fancypagestyle{plain}{%
\fancyhead{} % get rid of headers
\renewcommand{\headrulewidth}{0pt} % and the line
}


%%%%

\lstset{
	language=C++,
%basicstyle=\footnotesize\ttfamily\fontfamily{pcr}\selectfont,
%commentstyle=\footnotesize\ttfamily\fontfamily{pcr}\selectfont,
	columns=flexible,	
	basewidth=0.6em,
    %commentstyle=\footnotesize\Couriernew,
	basicstyle=\footnotesize\couriernew,
	numbers=left,                   % where to put the line-numbers
	numberstyle=\footnotesize,      % the size of the fonts that are used for the line-numbers
	stepnumber=1,
	showspaces=false,               % show spaces adding particular underscores
	showstringspaces=false, 
	tabsize=2,
	frame=single,
	extendedchars=true,
	breaklines=true,
	inputencoding=utf8x
  commentstyle=\color{blue}
}

% 修改 Title Formats
\usetikzlibrary{calc}

\newcommand{\ChapterEnglish}{}
\newcommand{\ChapterChinese}{}
\renewcommand{\chaptername}{Lecture}
%%
%% Create Chapter Image
%%
\newcommand{\ChapterImageParameter}{\includegraphics*{template/c.png}}
\titleformat{\chapter}[display]
{\gdef\chapterlabel{}\Huge\bfseries\helvetica}{\gdef\chapterlabel{\chaptertitlename\ \thechapter}}{0pt}{
\begin{tikzpicture}[remember picture, overlay]%
%%
%% The Lower Rectangle
%%
\fill[fill=black!90!white] (-10pt,-6pt) rectangle (\paperwidth, {3cm-6pt});%
%%
%% The Upper Rectangle
%%
\fill[fill=black!70!white] (-26pt,0) rectangle (\paperwidth, 3cm);%
%%
%% Right Image
%%
\node[anchor=south east,xshift=2.5cm,yshift=-8pt] at (\textwidth,0) {\ChapterImageParameter};%
\fill[fill=black!70!white,opacity=.6] (-26pt,0) rectangle (\paperwidth, 3cm);%
%%
%% Chapter Label
%%
\node[anchor=west,yshift=60pt,xshift=-16pt] {\color{white}\chapterlabel};%
%%
%% Chinese Title
%%
\node(ChineseTitle)[anchor=west,xshift=12pt,yshift={\baselineskip-8pt}] {\color{white}\ChapterChinese};%
%%
%% Middle Bar
%%
\node(MidBar)[minimum height=26pt,outer sep=0,inner sep=0,fill=black!10!white,xshift=4pt] at (ChineseTitle.east) {\color{black!10!white}\,};%
%%
%% English Title
%%
\node(EnglishTitle)
[anchor=west,yshift=-4pt] at (MidBar.east) {\color[rgb]{0.9,0.9,0.9}{\Large{\ChapterEnglish}}};%
\end{tikzpicture}%
}

\titlespacing*{\chapter}{0pt}{60pt}{40pt}

\titleformat{\section}[display]
{\gdef\sectionlabel{}\normalfont\Large\bfseries\helvetica}{\gdef\sectionlabel{\thesection}}{1em}{\begin{tikzpicture}[remember picture, overlay]%
\node (SecSymb) [anchor=south west,yshift={8pt-2pt},xshift={-21pt+2pt},fill=black!90!white] {\color{black!90!white}\sectionlabel};%
\node (SecSymb) [anchor=south west,yshift=8pt,xshift=-21pt,fill=black!70!white] {\color{white}\sectionlabel};%
\node[anchor=west,xshift=4pt,yshift=-1pt] at (SecSymb.east) {#1};%
\end{tikzpicture}}

\titlespacing*{\section}{0pt}{-20pt}{0pt}


\providecommand\NewChapter[2]{
	\renewcommand{\ChapterEnglish}{#2}
	\renewcommand{\ChapterChinese}{#1}
	\chapter{#1}
}


\usepackage[framemethod=TikZ]{mdframed}
\newcounter{exple}[chapter]
\makeatletter
\newcommand{\theexample}{\helvetica{例題~\arabic{chapter}-\arabic{exple}}}
\def\mdf@@sampletitle{}
\define@key{mdf}{title}{%
	\def\mdf@@sampletitle{#1}
}
\def\mdf@@samplesource{}
\define@key{mdf}{source}{%
	\def\mdf@@samplesource{#1}
}
\tikzset{tikzprobL/.style={fill=black!20,rectangle,minimum height=1.2\baselineskip}}
\tikzset{tikzborder/.style={ultra thick,color=black!50}}
\mdfdefinestyle{sampleproblem}{%
	outermargin=+12pt,
	innermargin=+12pt,
	innerleftmargin=24pt,
	innerrightmargin=16pt,
	innerbottommargin=8pt,
	innertopmargin=30pt,
	splittopskip=16pt,
	splitbottomskip=8pt,
	skipabove=16pt,
	leftline=false,
	rightline=false,
	topline=false,
	bottomline=false,
	frametitlefont=\tmtTitleFont,
	settings={\global\stepcounter{exple}},
	singleextra={%
		\node(SampleMark)[tikzprobL,anchor=north west] at (P-|O) {~\mdf@frametitlefont{\theexample}\hbox{~}};%
		\node(Topo)[minimum height={1.2\baselineskip},anchor=west] at (SampleMark.east) {\mdf@frametitlefont\mdf@@sampletitle};%
		\node(Sourcer) [minimum height={1.2\baselineskip},anchor=north east,yshift=-4pt] at (P) {\footnotesize\mdf@frametitlefont\mdf@@samplesource};%
		\path let
				\p1 = (Sourcer.south west),
				\p2 = (SampleMark.south east),
				\p3 = (Sourcer.north west)
			in
				coordinate (c1) at ($(\x1,\y2)+(-1.8\baselineskip ,0)$)
				coordinate (c2) at ($(\p3)+(-0.6\baselineskip, 0)$);%
		\draw[ultra thick,color=black!50] (SampleMark.south west) -- (c1) -- (c2) -- (Sourcer.north east);%
		\draw[ultra thick,color=black!50] (O) -- (O-|P);%
	},
	firstextra={\node(SampleMark)[tikzprobL,anchor=north west] at (P-|O)%
		{~\mdf@frametitlefont{\theexample}\hbox{~}};%
		\node(Topo)[minimum height={1.2\baselineskip},anchor=west] at (SampleMark.east) {\mdf@frametitlefont\mdf@@sampletitle};%
		\node(Sourcer) [minimum height={1.2\baselineskip},anchor=north east,yshift=-4pt] at (P) {\footnotesize\mdf@frametitlefont\mdf@@samplesource};%
		\path
			let
				\p1 = (Sourcer.south west),
				\p2 = (SampleMark.south east),
				\p3 = (Sourcer.north west)
			in
				coordinate (c1) at ($(\x1,\y2)+(-1.8\baselineskip ,0)$)
				coordinate (c2) at ($(\p3)+(-0.6\baselineskip, 0)$);
		\draw[tikzborder] (SampleMark.south west) -- (c1) -- (c2) -- (Sourcer.north east);%
		\draw[tikzborder] (O) -- (O-|P);%
		\node[xshift=-5pt, yshift=2pt, anchor=north,regular polygon, fill=black, regular polygon sides=3, inner sep=2pt,rotate=180] at (O-|P) {};%
	},
	middleextra={
		\draw[tikzborder] (P) -- (P-|O);
		\node[xshift=-5pt, yshift=-4pt, anchor=north,regular polygon, fill=black, regular polygon sides=3, inner sep=2pt] at (P) {};
		\draw[tikzborder] (O) -- (O-|P);
		\node[xshift=-5pt, yshift=2pt, anchor=north,regular polygon, fill=black, regular polygon sides=3, inner sep=2pt,rotate=180] at (O-|P) {};
	},
	secondextra={
		\draw[tikzborder] (P) -- (P-|O);
		\node[xshift=-5pt, yshift=-4pt, anchor=north,regular polygon, fill=black, regular polygon sides=3, inner sep=2pt] at (P) {};
		\draw[tikzborder] (O) -- (O-|P);
	},
	font={\Kai\mathversion{mymath}},
}
\makeatother
\providecommand\SampleProblem[3][經典問題]{
\begin{mdframed}[style=sampleproblem,title=#2,source=#1]%
%
#3%
\end{mdframed}
}

% kelvin added this part:
% He think the original form does not separate the sample problem itself well
% with following contents, and would like at least some simple line / box / vskip
% that distinguishes the sample problem from other content.
% It's ugly now, I know. Anyone is also welcomed to beautify this temporary setting.
%\providecommand\ExplainedSampleProblem[2]{
%
%\noindent\hrulefill
%
%{
%\leftskip 1em
%\rightskip 1em
%\noindent\textbf{(例題 x) {#1}}
%	
%	{#2}
%}
%
%\vspace{-0.5em}
%\noindent\hrulefill
%
%}

\newcounter{exeprob}[chapter]
\renewcommand{\theexeprob}{\helvetica{習題~\arabic{chapter}-\arabic{exeprob}}}
\providecommand\ExerciseProblem[2][經典問題]{%
\stepcounter{exeprob}%
\begin{tikzpicture}%
\fill[fill=black!10] (0,0) rectangle (\textwidth,-\baselineskip);%
\node(ProbNum)[anchor=north west] at (0,0) {\bfseries\helvetica\theexeprob};%
\path let \p1=(ProbNum.east) in coordinate (City) at ($(\x1,-\baselineskip)$);%
\fill[fill=black!30] (0,0) rectangle (City);%
\node(ProbNum)[anchor=north west] at (0,0) {\bfseries\helvetica\theexeprob};%
\node(PSource)[anchor=north east] at (\textwidth,0) {\footnotesize\bfseries\helvetica{#1}};%
\path
	let
		\p1 = (PSource.north west),
		\p2 = (\textwidth,-\baselineskip)
	in
		coordinate (c1) at ($(\x1-\baselineskip,\y2)$)
		coordinate (c2) at ($(\p1)$)
		coordinate (c3) at ($(\x2,0)$)
		coordinate (c4) at ($(\p2)$);%
\fill[fill=black!20] (c1) -- (c2) -- (c3) -- (c4) -- cycle;%
\node(PSource)[anchor=north east] at (\textwidth,0) {\footnotesize\tmtTitleFont{#1}};%
\node[anchor=west] at (ProbNum.east) {\tmtTitleFont{#2}};%
\end{tikzpicture}%
}

%%%%%%%%%%%%%%%%%%%%%%%%%%%%%%%%%%%%%%%%%%%%%%%%%%%%%%%%%%%%%%%%%%%%%
%% Pseudocode
%%%%%%%%%%%%%%%%%%%%%%%%%%%%%%%%%%%%%%%%%%%%%%%%%%%%%%%%%%%%%%%%%%%%%
%\usepackage[boxed,commentsnumbered]{algorithm2e}
\usepackage[noline, linesnumbered, commentsnumbered, algoruled, titlenotnumbered]{algorithm2e}

%%%%%%%%%%%%%%%%%%%%%%%%%%%%%%%%%%%%%%%%%%%%%%%%%%%%%%%%%%%%%%%%%%%%%
%% Sample Code Style
%%%%%%%%%%%%%%%%%%%%%%%%%%%%%%%%%%%%%%%%%%%%%%%%%%%%%%%%%%%%%%%%%%%%%
\usepackage[misc]{ifsym}
\newcounter{codepiece}[chapter]
\newcommand{\thecodepiecep}{\helvetica{程式碼片段~\arabic{chapter}-\arabic{codepiece}}}
\tikzset{tikzCodeBanner/.style={fill=black!10,rectangle,minimum height=\baselineskip}}
\tikzset{tikzCodeBorder/.style={line width=1.2pt,color=black!30}}
\makeatletter
\mdfdefinestyle{mdfsamplecode}{%
	outermargin=+12pt,
	innermargin=+12pt,
	innerleftmargin=12pt,
	innerrightmargin=0pt,
	innerbottommargin=-8pt,
	innertopmargin=\baselineskip,
	splittopskip=12pt,
	splitbottomskip=2pt,
	skipabove=16pt,
	leftline=false,
	rightline=false,
	topline=false,
	bottomline=false,
	settings={\stepcounter{codepiece}},
	singleextra={%
		%% Define Positions
		\node(NumberPart)[anchor=north east] at (P) {\footnotesize\bfseries\thecodepiecep};
		\node(CaptionPart)[anchor=north west] at (P-|O) {\small\tmtTitleFont\mdf@@sampletitle};
		%% Draw Background
		\fill[fill=black!10] (CaptionPart.south west) rectangle (NumberPart.north east); 
		%% Draw Text
		\node(NumberPart)[anchor=north east] at (P) {\footnotesize\bfseries\thecodepiecep};
		\node(CaptionPart)[anchor=north west] at (P-|O) {\small\tmtTitleFont\mdf@@sampletitle};
		%% The bottom line
		\draw[tikzCodeBorder] (O) -- (O-|P);
	},
	firstextra={
		%% Define Positions
		\node(NumberPart)[anchor=north east] at (P) {\footnotesize\bfseries\thecodepiecep};
		\node(CaptionPart)[anchor=north west] at (P-|O) {\small\tmtTitleFont\mdf@@sampletitle};
		%% Draw Background
		\fill[fill=black!10] (CaptionPart.south west) rectangle (NumberPart.north east); 
		%% Draw Text
		\node(NumberPart)[anchor=north east] at (P) {\footnotesize\bfseries\thecodepiecep};
		\node(CaptionPart)[anchor=north west] at (P-|O) {\small\tmtTitleFont\mdf@@sampletitle};
		\draw[tikzCodeBorder] (O) -- (O-|P);
		\node[xshift=-5pt, yshift=2pt, anchor=north,regular polygon, fill=black, regular polygon sides=3, inner sep=2pt,rotate=180] at (O-|P) {};
	},
	middleextra={
		\draw[tikzCodeBorder] (P) -- (P-|O);
		\node[xshift=-5pt, yshift=-4pt, anchor=north,regular polygon, fill=black, regular polygon sides=3, inner sep=2pt] at (P) {};
		\draw[tikzCodeBorder] (O) -- (O-|P);
		\node[xshift=-5pt, yshift=2pt, anchor=north,regular polygon, fill=black, regular polygon sides=3, inner sep=2pt,rotate=180] at (O-|P) {};
	},
	secondextra={
		\draw[tikzCodeBorder] (P) -- (P-|O);
		\node[xshift=-5pt, yshift=-4pt, anchor=north,regular polygon, fill=black, regular polygon sides=3, inner sep=2pt] at (P) {};
		\draw[tikzCodeBorder] (O) -- (O-|P);
	}
	%font={\Kai\mathversion{mymath}}
}
\makeatother




\providecommand\InlineCode\lstinline


\lstdefinestyle{ioicamp}{
	%numberstyle=\footnotesize\Garamond,
	numberstyle=\scriptsize\textifsym,
	frameshape={NNNNNNNNN}{nyn}{nnn}{NNNNNNNNN},
	framerule=1.2pt,
	rulecolor=\color[rgb]{0.7,0.7,0.7},
	belowskip=0pt
}
\lstset{style=ioicamp}

\providecommand\SourceCode[4][Default Caption]{
\begin{mdframed}[style=mdfsamplecode,title=#1,source=#2]%
\lstinputlisting[language=C++, firstline=#3, lastline=#4, firstnumber=1]{ref-codes/#2}\end{mdframed}
}

%%%
%%%     Figure
%%%
\usepackage{caption}

\makeatletter
\newcommand{\@InsertFigureWithFigure}[3]{
\begin{figure}[h]
\captionsetup{width=0.8\textwidth}
\centering
% hack by Darkpi
\ifdefined \CurrentPathName
\includegraphics[width={#2}]{\CurrentPathName/figures/#1}
\else
\includegraphics[width={#2}]{figures/#1}
\fi
\ifx&#3&
\else
\caption{#3}
\fi
\end{figure}
}
\newlength{\myskipwidth}
\newlength{\mypicwidth}
\newcommand{\@InsertFigureWithoutFigure}[3]{
\ifdefined \CurrentPathName
{%
\setlength{\mypicwidth}{#2}%
\setlength{\myskipwidth}{-0.5\mypicwidth}%
\addtolength{\myskipwidth}{0.5\textwidth}%
\addtolength{\myskipwidth}{-8pt}%
\hspace*{\myskipwidth}%
\includegraphics[width={#2}]{\CurrentPathName/figures/#1}}
\else
\setlength{\mypicwidth}{#2}%
\setlength{\myskipwidth}{-0.5\mypicwidth}%
\addtolength{\myskipwidth}{0.5\textwidth}%
\addtolength{\myskipwidth}{-8pt}%
\hspace*{\myskipwidth}%
\includegraphics[width={#2}]{figures/#1}%
\fi
}


\def\@myenvname{mdframed}
\providecommand\Figure[3]{%
\typeout{CURRENTVIR = [\@currenvir]\@myenvname}
\ifx\@currenvir\@myenvname
\expandafter\@InsertFigureWithoutFigure{#1}{#2}{#3}
\else
\expandafter\@InsertFigureWithFigure{#1}{#2}{#3}
\fi
}
\makeatother

\usepackage{array}
%\usepackage{amsthm}
\usepackage[amsmath,amsthm,thmmarks]{ntheorem}
\usepackage{mathdots}

\makeatletter
\newtheoremstyle{customthm}
{\item[\hskip\labelsep \theorem@headerfont\helvetica 定理\theorem@separator]}% no optional argument
  {\item[\hskip\labelsep \theorem@headerfont\helvetica ##3:\theorem@separator]}% optional argument
\makeatother
\theoremstyle{customthm}

\theoremheaderfont{\normalfont\bfseries}
\theorembodyfont{\normalfont\Kai}
\newmdtheoremenv[%
	outerlinewidth=1.2pt,%
	leftmargin=40pt,%
	rightmargin=40pt,%
	backgroundcolor=black!5,%
	outerlinecolor=black!60,%
	ntheorem=true,%
	font=\Kai\mathversion{mymath},%
]{theorem}{}[chapter]


%%%
%%%		Table Of Contents
%%%
%\usepackage{tocloft}
\usepackage{setspace}
\setcounter{tocdepth}{2}
\addtocontents{toc}{\protect\setstretch{0.7}}
%\setlength\cftparskip{0pt}
%\setlength\cftbeforechapskip{12pt}


%%%
%%%    皮皮的stuff (從game-section-5搬過來)
%%%
\usetikzlibrary{arrows}
\providecommand\DrawTree[1]{
\begin{tikzpicture}[->,>=stealth',level/.style={sibling distance = 0.7cm, level distance = 0.8cm},treenode/.style = {inner sep=0pt},arn_b/.style = {treenode, circle, white, draw=black, fill=black, text width=1.2em},arn_w/.style = {treenode, circle, white, draw=black, text width=1.2em}]
#1
\end{tikzpicture}
}

%%%
%%%    江誠敏的stuff
%%%
\usepackage{listings}
\usepackage{fancyvrb}
\usepackage{float,amssymb}
\usetikzlibrary{decorations.pathreplacing}
\usetikzlibrary{automata}
\newtheoremstyle{simpletheoremstyle}% 自定義Style
{\item[\hskip\labelsep  {\bf ##1}]}{}%       上下間距
\theoremstyle{simpletheoremstyle}
\newmdtheoremenv[%
	backgroundcolor=black!5,%
	ntheorem=true,%
  font=\Kai\mathversion{mymath},%
  leftline=false,topline=false,rightline=false,bottomline=false
  ]{lemma}{引理}[chapter]
\newmdtheoremenv[%
	backgroundcolor=black!5,%
	ntheorem=true,%
  font=\Kai\mathversion{mymath},%
  leftline=false,topline=false,rightline=false,bottomline=false
  ]{definition}{定義}
\newtheoremstyle{simpletheoremstyle2}% 自定義Style
{\item[\hskip\labelsep  {\it ##1}] \LiHeiPro}{}%       上下間距
\theoremstyle{simpletheoremstyle2}
\newtheorem{proof}{Proof.}
\theoremstyle{customthm}
%\newcommand*{\qed}{\hfill\ensuremath{\blacksquare}}
\newcommand{\contradict}{\ensuremath{\Rightarrow\hspace*{-1pt}\Leftarrow}}
\newcommand{\pass}{\mbox{}\vspace*{-10pt}}

\usepackage{wasysym} % add by seanwu
